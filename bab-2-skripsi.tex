\chapter{Pembahasan}
\section{Pengertian Wawasan Nusantara}
Wawasan Nusantara adalah pandangan yang esensial bagi sebuah bangsa di tengah lingkungan strategis yang cepat berubah, untuk memastikan kelangsungan eksistensi dan bertahan. Terminologi wawasan Nusantara menggambarkan pandangan suatu negara tentang dirinya dan lingkungannya yang dinamis, dengan mempertimbangkan beragam aspek seperti budaya, sejarah, geografi, ruang hidup, idealisme, falsafah negara, konstitusi, aspirasi, identitas, integritas, serta kemampuan dan daya saingnya.

Wawasan Nusantara berkembang dengan mengikuti perkembangan pemikiran dan penelitian untuk menjadi "Wawasan Nasional" yang merangkul seluruh aspek kehidupan nasional, termasuk ketahanan nasional. Kehidupan nasional harus dijaga agar tidak stagnan, karena kelangsungan hidup sangat penting. Oleh karena itu, diperlukan konsepsi Wawasan Nusantara yang mencakup seluruh aspek kehidupan nasional.

Wawasan Nusantara adalah geopolitik Indonesia yang mempertimbangkan baik aspek internal maupun eksternal. Ini mencakup konsep kelangsungan hidup negara, wilayah, masyarakat, tata kelola pemerintahan, dan jaminan keberlanjutan hidup negara. Untuk mencapai tujuan ini, lima aspek integrasi harus dipegang teguh, termasuk kesatuan wilayah, bangsa, budaya, ekonomi, dan pertahanan. Wawasan Nusantara adalah pandangan komprehensif tentang bagaimana memastikan kelangsungan hidup dan kemajuan nasional Indonesia.

\section{Mengenal Indonesia Dengan Wawasan Nusantara}

Mengenal Indonesia dengan Wawasan Nusantara memberikan pemahaman yang lebih mendalam tentang negara Indonesia dan bagaimana pandangan ini memengaruhi berbagai aspek kehidupan nasional. Kita dapat menjelaskan bagaimana Wawasan Nusantara memengaruhi berbagai aspek Indonesia seperti geografi, budaya, ekonomi, pertahanan, dan kesatuan nasional. 

\begin{itemize}
  \item \textbf{Konteks Geografis} \\ Indonesia terletak di antara dua samudra dan memiliki ribuan pulau. Dengan pandangan Wawasan Nusantara, Indonesia mengakui pentingnya wilayah maritimnya. Ini mencakup pemahaman tentang pentingnya laut sebagai jalur perdagangan, serta sumber daya alam yang dimiliki Indonesia di perairan tersebut.
  
  \item \textbf{Keragaman Budaya} \\ Indonesia adalah negara yang sangat beragam budaya, dengan ratusan kelompok etnis dan bahasa yang berbeda. Wawasan Nusantara mempromosikan kesatuan dalam keragaman, mengakui pentingnya melestarikan dan menghargai budaya-budaya ini sebagai bagian dari identitas nasional.  
  
  \item \textbf{Pertumbuhan Ekonomi} \\ Wawasan Nusantara juga mempertimbangkan aspek ekonomi. Dalam mengenal Indonesia dengan Wawasan Nusantara, kita dapat menjelaskan bagaimana strategi ekonomi Indonesia harus mempertimbangkan posisi geografisnya, sumber daya alam, dan hubungan perdagangan regional dan internasional.  
  
  \item \textbf{Pertahanan dan Keamanan} \\  Pertimbangan pertahanan adalah aspek penting dari Wawasan Nusantara. Ini mencakup bagaimana Indonesia melindungi wilayahnya dan menjaga perdamaian dan stabilitas di kawasan tersebut. Pandangan ini memainkan peran kunci dalam kebijakan pertahanan nasional.
  
  \item \textbf{Kesatuan Nasional} \\ Wawasan Nusantara juga menekankan pentingnya kesatuan nasional. Dalam mengenal Indonesia dengan Wawasan Nusantara, kita dapat menjelaskan bagaimana negara ini berusaha untuk memelihara persatuan di tengah keragaman dan tantangan yang ada.
  
\end{itemize}
