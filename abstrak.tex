Wawasan Nusantara adalah konsep dasar yang mendorong pemahaman dan kesadaran terhadap keberagaman budaya, geografi, dan sumber daya alam yang melimpah di Indonesia, dikenal sebagai Nusantara. Konsep ini mendukung persatuan, kedaulatan, dan integritas wilayah, serta menjalin hubungan harmonis dengan negara-negara tetangga. Wawasan Nusantara menjadi landasan kebijakan luar negeri dan visi pembangunan nasional Indonesia. Pemahaman konsep ini memperkuat nasionalisme, menghormati keragaman budaya, mendorong kerja sama antar-masyarakat, dan pelestarian lingkungan. Meskipun penting, masih banyak yang belum memahami konsep ini, menciptakan tantangan dalam mencapai tujuan kesatuan, kedaulatan, dan pembangunan nasional yang diamanatkan oleh Wawasan Nusantara.

