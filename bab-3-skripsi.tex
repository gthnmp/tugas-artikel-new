\chapter{PENUTUP}

\section{Kesimpulan}
Wawasan Nusantara memiliki peran penting dalam memastikan kelangsungan eksistensi dan kesejahteraan Indonesia. Konsep ini mencakup lima aspek integrasi yang mencakup wilayah, bangsa, budaya, ekonomi, dan pertahanan. Dalam mengenal Indonesia dengan Wawasan Nusantara, kita memahami betapa esensialnya pandangan ini dalam mengelola sumber daya alam, melindungi wilayah, mempromosikan persatuan dalam keragaman budaya, dan menggerakkan pertumbuhan ekonomi. Oleh karena itu, penting bagi semua warga Indonesia untuk memahami dan menerapkan konsep Wawasan Nusantara dalam berbagai aspek kehidupan, sehingga negara ini dapat tetap kuat, berkelanjutan, dan bersatu di tengah perubahan yang cepat.

\section{Saran}
\begin{itemize}
  \item \textbf{Pendidikan:} Mendorong pendidikan yang lebih mendalam tentang Wawasan Nusantara di sekolah-sekolah dan lembaga-lembaga pendidikan tinggi. Ini akan membantu generasi muda memahami konsep ini secara lebih baik.
  
  \item \textbf{Kampanye Kesadaran:} Mengadakan kampanye kesadaran masyarakat tentang pentingnya Wawasan Nusantara melalui media sosial, seminar, dan program-program pendidikan masyarakat.
  
  \item \textbf{Pelestarian Budaya:} Mendukung upaya pelestarian budaya lokal di seluruh Indonesia sebagai bagian integral dari Wawasan Nusantara. Dukungan terhadap kesenian tradisional, bahasa, dan budaya lokal dapat memperkuat keragaman budaya.
  
  \item \textbf{Kesadaran Lingkungan:} Mengedukasi masyarakat tentang pentingnya menjaga lingkungan dan sumber daya alam di Indonesia sesuai dengan prinsip-prinsip Wawasan Nusantara.
  
  \item \textbf{Kerjasama Regional:} Mendorong kerjasama yang erat dengan negara-negara tetangga dan pemahaman yang lebih baik tentang peran Indonesia di kawasan Asia Tenggara.
  
  \item \textbf{Keterlibatan Masyarakat:} Mendorong partisipasi aktif masyarakat dalam kebijakan dan inisiatif yang mendukung Wawasan Nusantara, termasuk pemantauan terhadap upaya pelestarian sumber daya alam dan budaya.
\end{itemize}
